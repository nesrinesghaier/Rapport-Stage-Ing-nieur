\section*{Introduction générale}
%\section*{Introduction}
\addcontentsline{toc}{section}{Introduction générale}
%\markboth{Introduction générale}{} %To redefine the section page head
Un Formulaire est un très bon moyen de recueillir des données qui seront par la suite exploitables dans une multitude de cas: améliorer l’expérience utilisateur sur les sites, améliorer les produits et les services, avoir de nouvelles idées ...
La réalisation d'un tel questionnaire en ligne n'était pas toujours une chose aisée, avec le développement d’Internet, de nombreux outils ont vu le jour. Il existe à l’heure actuelle un grand nombre de logiciels de création de questionnaires en ligne mais chacun répondra à des attentes différentes.\\

Afin de répondre à des besoins précis et personnaliser ces outils informatiques d'une manière plus efficace pour l’amélioration au niveau de la qualité de ses services, l'entreprise SOFIA Technologies a proposé dans le cadre de ce projet la réalisation d'une application web qui permet de générer des formulaires dynamiques en ligne.\\    

Notre travail se traduit dans ce rapport qui développe les différentes phases par lesquelles nous sommes passées et qui sont organisées en trois chapitres de la manière suivante: \\
Nous débutons par présenter le cadre du projet, l’étude et la critique de l’existant, les objectifs
à atteindre, le langage et la méthodologie adoptés.\\
Le deuxième chapitre sera consacré à l’analyse et l’étude des besoins au cours duquel nous
allons énumérer les besoins fonctionnels et non fonctionnels, les intervenants au sein de notre
système et la conception de notre application.\\
La réalisation des sprints sera effectuée au niveau du dernier chapitre durant lequel nous allons détailler les différentes
fonctionnalités de l'application et les outils et les technologies utilisés.\\

Nous clôturerons ce rapport par une conclusion générale dans laquelle nous évaluerons le
travail réalisé au sein de la société et nous proposerons des perspectives dans le but d’améliorer
notre travail.
